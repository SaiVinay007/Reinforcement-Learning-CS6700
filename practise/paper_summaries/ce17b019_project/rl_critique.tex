\documentclass{article}

% if you need to pass options to natbib, use, e.g.:
%     \PassOptionsToPackage{numbers, compress}{natbib}
% before loading neurips_2018

% ready for submission
% \usepackage{neurips_2018}

% to compile a preprint version, e.g., for submission to arXiv, add add the
% [preprint] option:
%     \usepackage[preprint]{neurips_2018}

% to compile a camera-ready version, add the [final] option, e.g.:
\usepackage[final]{nips_2018}

% to avoid loading the natbib package, add option nonatbib:
%     \usepackage[nonatbib]{neurips_2018}

\usepackage[utf8]{inputenc} % allow utf-8 input
\usepackage[T1]{fontenc}    % use 8-bit T1 fonts
\usepackage{hyperref}       % hyperlinks
\usepackage{url}            % simple URL typesetting
\usepackage{booktabs}       % professional-quality tables
\usepackage{amsfonts}       % blackboard math symbols
\usepackage{nicefrac}       % compact symbols for 1/2, etc.
\usepackage{microtype}      % microtypography

\title{ Reinforcement learning ( CS6700 ) \\
Paper critique report -  \\
Convolutional Neural Networks on Graphs
with Fast Localized Spectral Filtering}

% The \author macro works with any number of authors. There are two commands
% used to separate the names and addresses of multiple authors: \And and \AND.
%
% Using \And between authors leaves it to LaTeX to determine where to break the
% lines. Using \AND forces a line break at that point. So, if LaTeX puts 3 of 4
% authors names on the first line, and the last on the second line, try using
% \AND instead of \And before the third author name.

\author{%
  Sai Vinay G \\
  CE17B019\\
  \\
  28th Jan. 2019 \\
  % \texttt{hippo@cs.cranberry-lemon.edu} \\
  % examples of more authors
  % \And
  % Coauthor \\
  % Affiliation \\
  % Address \\
  % \texttt{email} \\
  % \AND
  % Coauthor \\
  % Affiliation \\
  % Address \\
  % \texttt{email} \\
  % \And
  % Coauthor \\
  % Affiliation \\
  % Address \\
  % \texttt{email} \\
  % \And
  % Coauthor \\
  % Affiliation \\
  % Address \\
  % \texttt{email} \\
}

\begin{document}
% \nipsfinalcopy is no longer used

\maketitle


\section{Summary}

This paper presents a way to extend convolutional neural networks to represent 
graph-structured data such as social networks, brain connectomes or words’ embedding and 
assuming the data is locally stationary. To perform convolutions and pooling operators 
on graphs is not straightforward as the operations are only defined on regular grids
and one more thing is to make them fast (linear complexity). \\

\subsection{Learning fast localized spectral filters} 

\begin{itemize}
  \item {\bf Graph Fourier Transform} : Graph laplacian is a real symmetric positive semidefinite matrix, 
  it has an complete set of orthonormal eigenvectors , known as the graph Fourier modes, and
  their associated ordered real nonnegative eigenvalues  , identified as the frequencies of the
  graph. The Laplacian is indeed diagonalized by the Fourier basis
  The graph Fourier transform of a signal is then defined as and its inverse as As on Euclidean
  spaces, that transform enables the formulation of fundamental operations such as filtering.
  \item {\bf Spectral filtering of graph signals} :
  \item {\bf Polynomial parametrization for localized filters} :
  \item {\bf Learning filters} :
  
  \begin{itemize}
    \item[-] Graph laplacian (L) :
     
    $L = D - W $ $ \in \mathbb{R}^{n\times n} $   - combinatorial \\
    $L = {I_n} - D^{- \frac{1}{2}}WD^{- \frac{1}{2}} $  - normalized
  \end{itemize} 

\end{itemize}

\subsection{Graph Coarsening}
Pooling operation requires meaningful neighborhoods on graphs, where similar vertices are
clustered together and for multiple layers requires multi-scale clustering.
Multilevel clustering algorithms where each level produces a coarser graph which corresponds to the data domain seen at a different resolution.
Coarsing phase of Graclus multilevel clustering algorithm, which has been shown to be extremely ef-
ficient at clustering a large variety of graphs has been used.

\subsection{Fast Pooling of Graph Signals}

\subsection{Experiments} 


\section{Criticism}
  \begin{itemize}
    \item 
    \item[-]  The paper has achieved convolution operations on graphs with time complexity similar to that of Classical ConvNets which is quite brilliant.
    \item[-]  It provides us new insights to extend the classical ConvNets to more general and vast field of social networks still capable of running
    on the present GPUs   

   
  \end{itemize} 



\section*{References}

\small

[1] Michaël Defferrard, Xavier Bresson, Pierre Vandergheynst; arXiv:1606.09375.

[2] Xavier Bresson: "Convolutional Neural Networks on Graphs" [Youtube].


\end{document}
